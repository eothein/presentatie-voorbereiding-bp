%%----------------------------------------------------------------------------
%% Presentatie HoGent Bedrijf en Organisatie
%%----------------------------------------------------------------------------
%% Auteur: Bert Van Vreckem [bert.vanvreckem@hogent.be]

\documentclass{beamer}

%==============================================================================
% Aanloop
%==============================================================================

%---------- Packages ----------------------------------------------------------

\usepackage{graphicx,multicol}
\usepackage{comment,enumerate,hyperref}
\usepackage{amsmath,amsfonts,amssymb}
\usepackage{tikz}
\usepackage[dutch]{babel}
\usepackage[utf8]{inputenc}
\usepackage{multirow}
\usepackage{eurosym}
\usepackage{listings}
\usepackage[T1]{fontenc}
\usepackage{lmodern}
\usepackage{textcomp}

%---------- Configuratie ------------------------------------------------------

\usetikzlibrary{arrows,shapes,backgrounds,positioning,shadows}

\usetheme{hogent}

%---------- Commando-definities -----------------------------------------------

\newcommand{\vrstlsemeen}{29/09/2017}
\newcommand{\vrstlsemtwee}{08/12/2017}
\newcommand{\feedsemeen}{06/10/2017}
\newcommand{\feedsemtwee}{22/12/2017}

\newcommand{\deadlineeen}{08/01/2018}
\newcommand{\deadlinetwee}{01/06/2018}
\newcommand{\deadlinedrie}{20/08/2018}


%---------- Info over de presentatie ------------------------------------------



\title[BP 2017 - 2018]{Bachelorproef academiejaar 2017 2018}
\author{Jens {Buysse} \small(\href{mailto:jens.buysse@hogent.be}{jens.buysse@hogent.be})}
\date{\today}

%==============================================================================
% Inhoud presentatie
%==============================================================================

\begin{document}

%---------- Front matter ------------------------------------------------------

% Dia met het HoGent logo
\HoGentLogo

% Titeldia met faculteitslogo
\begin{frame}[plain]
  \titlepage
\end{frame}

\begin{frame}
  \frametitle{Inhoud}

  \tableofcontents
\end{frame}

%---------- Inhoud ------------------------------------------------------------

% Dia voor sectiekop, voorbeeld met een afbeelding onderaan de pagina
\section{Bachelorproef}
\sectionframe{%
  Doelstellingen \& inhoud
  \vfill
  \includegraphics[width=3cm]{img/HG-woordmerk-inv}
}

\begin{frame}{Eindcompetenties}
\begin{alertblock}{}
	De student kan bestaande en innovatieve \textcolor{HoGentAccent6}{(IT-)oplossingen} op een \textcolor{HoGentAccent6}{methodologisch correcte manier} kritisch \textcolor{HoGentAccent6}{onderzoeken}, \textcolor{HoGentAccent6}{evalueren} en \textcolor{HoGentAccent6}{(her)ontwerpen of optimaliseren}
\end{alertblock}
\end{frame}

\begin{frame}{Doelstellingen}
	\begin{itemize}
		\item Kan een concrete \textcolor{HoGentAccent1}{doelstelling} en \textcolor{HoGentAccent1}{onderzoeksvragen} voor een eigen onderzoeksonderwerp formuleren en verduidelijken 
		\item Kan de data \textcolor{HoGentAccent1}{methodologisch verantwoord} verzamelen 
		\item Kan \textcolor{HoGentAccent1}{methodologisch verantwoorde} analyses maken van de verzamelde data 
		\item Kan de onderzoeksvraag onderbouwd beantwoorden a.d.h.v. correcte \textcolor{HoGentAccent1}{methodologische analyses} van de verzamelde data en door verschillende alternatieve oplossingen te evalueren 
		\item Kan in functie van de onderzoeksvragen geschikte \textcolor{HoGentAccent1}{vakliteratuur} evalueren, selecteren en verwerken in een \textcolor{HoGentAccent1}{literatuurstudie} 
		\item Kan een \textcolor{HoGentAccent1}{gestructureerd wetenschappelijk document} schrijven, voorzien van referenties en conform de aangereikte template
	\end{itemize}
\end{frame}

\section{Organisatie}
\sectionframe{}
\subsection{Ondersteuning \& Opvolging}

\begin{frame}{Ondersteuning \& Opvolging}
\begin{description}
	\item[Promotor] Is een lector informatica die het proces en opvolging ondersteunt. Dit is je eerste lijn bij vragen en problemen.
	\item[Co-promotor] Is een vakexpert (niet noodzakelijk @Hogent) en helpt je inhoudelijk. \textcolor{HoGentAccent1}{Deze co-promotor moet je zelf zoeken.}
	\item[Bachelorproefco\"ordinator] Co\"ordineert de algemene werking van de bachelorproeven. Heb je in principe geen direct contact mee.
\end{description}
\end{frame}



\subsection{Proces}
\begin{frame}{Workflow}
	\begin{enumerate}
		\item Je zoekt een onderwerp en een co-promotor
		\item Je schrijft een bachelorproefvoorstel volgens de richtlijnen
		\item Je dient het voorstel in
		\begin{itemize}
			\item semester 1 : \vrstlsemeen
			\item semester 2 : \vrstlsemtwee
		\end{itemize}
		\item  Je krijgt feedback op uw voorstel
		\begin{itemize}
			\item semester 1: \feedsemeen
			\item semester 2: \feedsemtwee
		\end{itemize} 
		\item Je maakt de bachelorproef volgens de richtlijnen
		\item Je dient uw bachelorproef in volgens de richtlijnen
		 \begin{itemize}
		 	\item semester 1: \deadlineeen
		 	\item semester 2: \deadlinetwee
		 	\item herexamen: \deadlinedrie
		 \end{itemize}
	 	\item Je verdedigt uw BP voor een jury 
	\end{enumerate}
\end{frame}

\section{Een BP-onderwerp}
\sectionframe{}
\begin{frame}{Een onderwerp kiezen}
	\begin{itemize}
		\item Studie van een ICT-gerelateerd onderwerp (bv. via je stage, via vak onderzoekstechnieken)
		\item Voldoende uitdagend, maar realistisch
		\item Start vanuit concreet probleem/vraag uit het werkveld
		\item Gaat verder dan het verzamelen van informatie
		\item Nuttig voor anderen in je vakgebied
		\item Origineel en individueel
	\end{itemize}
\end{frame}

\begin{frame}{Niet-aanvaardbaar \dots}
	\begin{itemize}
		\item Zuivere literatuurstudie
		\item Programmeerproject
		\item Vergelijking van frameworks, producten, services zonder
		concrete case, zonder requirementsanalyse
		\item  Hergebruik resultaten stage
		\item  \dots
	\end{itemize}
\end{frame}

\begin{frame}{Inspiratie nodig ?}
	\begin{itemize}
		\item Stagebedrijf
		\begin{itemize}
		\item maar losstaand van stageopdracht!
		\item stageopdracht = uitvoerend
		\item bachelorproef = onderzoekend
	\end{itemize}
		\item Lijst op Chamilo
		\item Video's van conferenties in je vakgebied
		\item Vakliteratuur (incl. blogs)
		\item Passend binnen eigen toekomstplannen?
		\item Lectoren specialisatievakken?
	\end{itemize}
\end{frame}

\begin{frame}{BP voorstel}
	Zie template !
\end{frame}

\begin{frame}{Enkele tips}
	\begin{itemize}
		\item Zoek een originele titel die beschrijft wat je onderzoekt. Niet het domein specifi\"eren. 
		\item Beschrijf dit “objectief” (geen “ik”, geen “opstel-stijl”)
		\begin{itemize}
			\item \textcolor{HoGentAccent2}{NIET}: “Ik vind cloud-computing heel interessant dus wil dit voor mijn
			bachelorproef nader bestuderen.”
			\item \textcolor{HoGentAccent3}{WEL}: “Firma X is op zoek naar een open source platform voor het
			opzetten van IaaS in eigen beheer.”
		\end{itemize}
	\end{itemize}
\end{frame}

\section{Indienen}
\sectionframe{}
\begin{frame}{Hoe indienen}
	Op \href{https://scriptie.hogent.be/}{https://scriptie.hogent.be/}
	\begin{enumerate}
		\item plagiaatcontrole
		\item $\geq 15/20$: ter beschikking via \href{bib.hogent.be}{bib.hogent.be}
\end{enumerate}

Op vraag van een van de partijen kan ook een papieren versie gevraagd worden (af te spreken met je promotor). 

\end{frame}

\begin{frame}{Ontvankelijkheid}
	Enkele vereisten (niet gelimiteerd tot )
	\begin{enumerate}
		\item Voldoet aan alle voorwaarden vernoemd in de BP richtlijnen.
		\item Minimumlengte (30 blz en 10K woorden)
		\item Scriptie-onderwerp was formeel goedgekeurd
		\item Doorstaat plagiaatcontrole
		\item Vormvereisten gerespecteerd (ook taal!)
		\item Referentielijst en verwijzingen volgens APA-stijl
		\item Consistente, strakke lay-out (LaTeX!)
		\item Voorpagina volgens sjabloon (scriptie.hogent.be)
	\end{enumerate}
\end{frame}

\begin{frame}{Beoordeling}
	Verschillende niveaus:
	\begin{itemize}
		\item onvoldoende (<10), voldoende ($\geq$10), goed ($\geq$13), zeer goed
		($\geq$15), uitstekend ($\geq17$), uitmuntend ($19$)
		\item Alle indicatoren voor een bepaald niveau moeten behaald
		zijn om het overeenkomstige examencijfer te behalen
	\end{itemize}

\end{frame}

\begin{frame}{Tot slot}
	Let op! We houden je niet meer bij het handje
	\begin{itemize}
	\item Resultaat Bachelorproef is jouw verantwoordelijkheid
	\item Jij moet ons overtuigen van wat je kan
	\item Laat dit niet liggen tot na je stage...
	\item Lees de richtlijnen!
		\end{itemize}
\end{frame}
%---------- Back matter -------------------------------------------------------

\end{document}
